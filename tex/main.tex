%!TeX program = pdflatex
% Folgert Karsdorp's Curriculum Vitae
% Email: folgert.karsdorp@meertens.knaw.nl
% Web: https://www.karsdorp.io/
% Repo: https://github.com/fbkarsdorp

\documentclass[12pt,letterpaper]{report}

\usepackage[T1]{fontenc} % output T1 font encoding (8-bit) for accented characters as single glyph
\usepackage[strict,autostyle]{csquotes} % smart and nestable quote marks
\usepackage[USenglish]{babel} % regionalize hyphens, quote marks, etc automatically
\usepackage{microtype}% improve text appearance with kerning, etc
\usepackage{datetime} % enable formatting of date output
\usepackage{tabto}    % make nice tabbing
\usepackage{hyperref} % enable hyperlinks and pdf metadata
\usepackage{geometry} % manually set page margins
\usepackage{enumitem} % enumerate with [resume] option
\usepackage{titlesec} % allow custom section fonts
\usepackage{setspace} % custom line spacing

% what is your name?
\newcommand{\myname}{Folgert Karsdorp}

% select default typefaces
\usepackage{ebgaramond} % document's serif typeface
\usepackage{helvet}     % document's sans serif typeface

% how far to tab for list items with left-aligned date: different fonts need different widths
\newcommand{\listtabwidth}{1.7cm}

% define font to use as document's title
\newcommand{\namefont}[1]{{\normalfont\bfseries\Huge{#1}}}

% set section heading fonts and before/after spacing
\SetTracking{encoding=*, family=\sfdefault}{30} % increase sans serif headings tracking
\titleformat{\section}{\lsstyle\sffamily\small\bfseries\uppercase}{}{}{}{}
\titlespacing{\section}{0pt}{30pt plus 4pt minus 4pt}{8pt plus 2pt minus 2pt}

% set subsection heading fonts and before/after spacing
\titleformat{\subsection}{\lsstyle\sffamily\footnotesize\bfseries}{}{}{}{}
\titlespacing{\subsection}{0pt}{16pt plus 4pt minus 4pt}{4pt plus 2pt minus 2pt}

% set page margins (assumes letter paper)
\geometry{body={6.5in, 9.0in},
    left=1.0in,
    top=1.0in}

% prevent paragraph indentation
\setlength\parindent{0em}

% set line spacing
\setstretch{0.9}

% define space between list items
\newcommand{\listitemspace}{0.25em}

% make unordered lists without bullets and use compact spacing
\renewenvironment{itemize}
{\begin{list}{}{\setlength{\leftmargin}{0em}
                \setlength{\parskip}{0em}
                \setlength{\itemsep}{\listitemspace}
                \setlength{\parsep}{\listitemspace}}}
{\end{list}}

% make tabbed lists so content is left-aligned next to years
\TabPositions{\listtabwidth}
\newlist{tablist}{description}{3}
\setlist[tablist]{leftmargin=\listtabwidth,
    labelindent=0em,
    topsep=0em,
    partopsep=0em,
    itemsep=\listitemspace,
    parsep=\listitemspace,
    font=\normalfont}

% print only the month and year when using \today
\newdateformat{monthyeardate}{\monthname[\THEMONTH] \THEYEAR}

% define hyperlink appearance and metadata for pdf properties
\hypersetup{
    colorlinks  = true,
    urlcolor    = black,
    pdfauthor   = {\myname},
    pdftitle    = {\myname: Curriculum Vitae},
    pdfsubject  = {Curriculum Vitae},
    pdfpagemode = UseNone
}

\usepackage[backend=biber, style=apa, maxcitenames=2, maxbibnames=99, sorting=ydnt,
            uniquelist=false, mincitenames=1, citestyle=authoryear]{biblatex}
\addbibresource{../static/references.bib}

\usepackage{authblk}


\begin{document}

    \raggedright{}

    % display your name as the document title
    \namefont{\myname}

    % affiliation and contact info blocks
    \vspace{1em}
    \begin{minipage}[t]{0.600\textwidth}
        % current primary affiliation, left-aligned
        Meertens Instituut \\
        Koninklijke Nederlandse Akademie van Wetenschappen
    \end{minipage}
    \begin{minipage}[t]{0.390\textwidth}
        % contact info details, right-aligned
        \flushright{}
        \href{mailto:folgert.karsdorp@meertens.knaw.nl}{folgert.karsdorp@meertens.knaw.nl} \\
        \href{https://www.karsdorp.io}{www.karsdorp.io}
    \end{minipage}

\section*{Education}
\begin{tablist}
\item[PhD] \tab{}\textit{Retelling Stories. A Computational-Evolutionary Perpective},
  (\textit{Cum Laude}), Radboud University, Nijmegen, 2012--2016
  \item[MA] \tab{}Linguistics (\textit{Cum Laude}), Leiden University, 2007--2009
  \item[BA]  \tab{}Dutch Language and Culture, Leiden University, 2004--2007
\end{tablist}


\section*{Academic Appointments}
\begin{tablist}
\item[2020--]   \tab{}KNAW Meertens Institute \\
  Senior researcher, Oral Culture Research Group \& Affiliated member of the KNAW Humanities
  Cluster DHlab

\item[2016--20] \tab{}KNAW Meertens Institute \\
  Tenure Track researcher, Oral Culture Research Group \& Affiliated member of the KNAW
  Humanities Cluster DHlab

\item[2016--16] \tab{}Radboud University \\
  Research assistant, supervision by Prof. Dr. Antal van den Bosch

\item[2010--11] \tab{}Instituut voor de Nederlandse Taal \\
  Researcher, (Formerly known as the Instituut voor Nederlandse Lexicologie)

\item[2009--10] \tab{}Free University Berlin \\
  Research Assistant, Supervision by Prof. Dr. Matthias Hüning
\end{tablist}

\section*{Research Areas}

I’m a researcher in Computational Humanities and Cultural Evolution at the
Meertens Institute of the Royal Netherlands Academy of Arts and Sciences
(Amsterdam, the Netherlands). My research focuses on cultural change, measuring
cultural diversity and compositional complexity, and how we can account for
biases in our estimations of diversity and complexity. To do that, I use
computational models from Machine Learning, Cultural Evolution, and Ecology.
Besides cultural change and diversity, I’m also interested in teaching about
computer programming in the context of the Humanities. I published a text book
with Princeton University Press about using Python for Humanities data analysis.

\nocite{*}
\printbibliography[title=Publications, heading=subbibliography]

\section*{Invited Talks}
\begin{tablist}
\item[2022] \tab{}``Forgotten Books The application of unseen species models to the
  survival of culture''. Invited talk at the CUDAN Open Lab Seminar series, Tallinn
  University, Tallinn, Estonia, September 19, 2022.
\item[2022] \tab{}``Forgotten Books The application of unseen species models to the
  survival of culture''. Keynote at the Summer School for Literary Studies and DH, Leiden
  University, Leiden, the Netherlands, June 20, 2022.
\item[2021] \tab{}``Unseen Species Models from Ecology to Estimate the Losses of Medieval
  Literature: Advances in an International Comparison''. Talk at the International
  Medieval Conference 2021 (IMC 2021) in the session “Loss and Transmission: Quantitative
  Approaches to Modelling the Dissemination and Survival of Medieval Literature”, 8 July
  2021
\item[2021] \tab{}``The Birds in the Bush. What can Occupancy Models from Ecology Teach us
  about the Survival of Medieval Literature?'', Invited presentation at the symposium The
  Human in Digital Humanities, June 23 2021. 
\item[2021] \tab{}``Libraries as book traps. Statistical methods from ecology to study the
  survival of historic literature'', Keynote lecture at the conference \textit{Old Books
    and New Technologies}: Medieval Books and the Digital Humanities in the Low Countries,
  6 May 2021
\item[2021] \tab{}``Estimating the loss of medieval literature with methods from
  ecology'', invited talk in the online lecture series Beyond the Patterns. 10 March 2021,
  FAU, Nürnberg, DE. 
\item[2021] \tab{}``Estimating the loss of medieval literature with methods from
    ecology'', invited talk in the ATNU/IES Virtual Speaker Series 2020/2021 \#6. 21
    February 2021, Newcastle, UK
\item[2020] \tab{}``Blind or directed? Cultural Evolution of Children's Books'',
    Children’s Literature and Digital Humanities, University of Antwerp, 22-23 October
    2020
\item[2020] \tab{}``Bias, diversity and survival. Can statistical methods from ecology
    estimate the loss of medieval literature?'' Online Árni Magnússon Birthday lecture, 13
    november 2020, The Arnamagnæan Institute, University of Copenhagen
\item[2020] \tab{}``Estimating the loss of medieval literature with unseen species model
    from ecodiversity''. Presentation at the online conference Dark Archives 20/20: A
    Voyage into the Medieval Unread and Unreadable, Tuesday 8th September 2020
\item[2019] \tab{}``Cultural entrenchment of folktales is encoded in language'',
  Presentation at the Lecture series of the Leiden University Centre for Digital
  Humanities, Leiden, 28 February 2019
\item[2019] \tab{}``The artificial synthesis of hiphop lyrics''. Humlab talk series. Umeå
  universitet, Sweden. 12 February 2019.
\item[2019] \tab{}``Keeping it real: the artificial synthesis of hiphop lyrics''. Vogin-IP,
  Amsterdam. 22 March 2019
\item[2018] \tab{}``Hoe verzin je het?''. Kenniscafe de Balie over wetenschap van
  creativiteit. April 16, 2018, Amsterdam. 
\item[2018] \tab{}``Cultural Evolution in Children’s Literature''. Cultural Evolution in
  Children's Literature, Jul 5, 2018. 
\item[2018] \tab{}``How to Read a Million Stories? Digital Text Analysis for the Study of
  Children's Literature''. Plenary lecture at Children’s Literature Summer School 2018,
  University of Antwerp, 5 July 2018
\item[2018] \tab{}``Synthesising humanities. Explaining complex models through simple data
  synthesis''. Closing keynote lecture at Digitial Humanities Benelux conference 2018.
  Amsterdam, The Netherlands, 8 June 2018
\item[2018] \tab{}``Cultural entrenchment of folktales is encoded in language'',
  Presentation at the workshop Interdisciplinary Workshop on Folk and Fairytales Digital,
  15 February 2018
\item[2018] \tab{}``Willekeur in de Geesteswetenschap''. De analyse van emoties en
  betekenis in tekst en beeld, 15 Jun 2018, VU Amsterdam.
\item[2017] \tab{}``Een literaire robot als schrijfhulp''. Robots: van hulpje tot
  kunstenaar (Research Files 4), Pakhuis de Zwijger, October 12, 2017. 
\item[2017] \tab{}``Character Bias in Transmitting Folktales''. 14th SIKS/Twente Seminar
  on Searching and Ranking. March 10, 2017. 
\end{tablist}

\section*{Open Data Sets}
\begin{tablist}
\item[2019] \tab{}\textit{Supplemental Materials for ``Humanities Data Analysis''}\\ Data
  discussed in the manuscript \textit{Humanities Data Analysis: Case Studies with Python}.
  Each folder in this dataset contains data used or discussed in one chapter. Most of the
  data are texts published before 1900. These texts are in the public domain. \\
  \url{https://doi.org/10.5281/zenodo.891264}
\item[2016] \tab{}\textit{Story network data sets} \\
  Data discussed in Karsdorp \& Van den Bosch (2016): The structure and evolution of story
  networks. Royal Society Open Science, 3, 160071.
  \url{https://doi.org/10.5281/zenodo.51588} 
\end{tablist}

\section*{Open-Source Software}
\begin{tablist}
  \item[2020--21] \tab{}Copia. Estimating the survival of cultural heritage artifacts with
    unseen species models from ecology. Copia is a Python package that can be used for
    estimating the survival of artifacts from cultural heritage, based on established
    unseen species models from ecology. \url{https://copia.readthedocs.io/en/latest/}
  \item[2018--19] \tab{}Deep flow. Open source code for the Hip Hop experiment executed at
    Lowlands 2018. \url{https://github.com/fbkarsdorp/deepflow}
  \item[2015--21] \tab{}Momfer. Meertens Motif Finder. Online application and app for
    browsing Thompson's \textit{Motif Index}. \url{https://momfer.meertens.knaw.nl/}
  \item[2013--19] \tab{}Open source and open access materials for an interactive course
    about Python in the Humanities. \url{https://github.com/fbkarsdorp/python-course}
\end{tablist}

\section*{Public Outreach Projects}
\begin{tablist}
\item[2018] \tab{}\textit{Deepflow: Linguistic models of authenticity judgments for
    artificially generated rap lyrics}\\
  Collaboration between the Meertens Institute, the University of Antwerp and Lowlands
  Science on generating artificial hip hop lyrics using neural language models. The
  project involved an experiment carried out in the context of a large, mainstream
  contemporary music festival in the Netherlands (Lowlands). We developed an app to study
  crowd-sourced authenticity judgments for such artificially generated texts.
\item[2017] \tab{}\textit{Asibot: Writing science fiction in a co-creative process}\\
  Collaboration between the Meertens Institute, the University of Antwerp, the CPNB
  foundation and writer Ronald Giphart on employing Artificial Intelligence for literary
  language generation. The result of this experiment was a new story, "De Robot van de
  Machine is de mens" – written by Giphart and a neural network language generation system
  – of which 250,000 copies were disseminated to the public during the national campaign
  "Nederland Leest" in November 2017.
\end{tablist}

\section*{Teaching (selected courses)}
\begin{tablist}
  \item[2022] \tab{}``Humanities Data Analsysis wit Python''. Digital Curriculums
    Inspirational Seminar, May 24 2022, Aarhus University, Denmark.
  \item[2022] \tab{}``Humanities Data Analysis with Python''. CUSO Winterschool, January 31 
    -- February 4 2022, University of Neuchâtel, Switserland. 
  \item[2021] \tab{}``Humanities Data Analysis with Python. Wetting one’s appetite with
    historic cookbooks''. Workshop at the Leiden Summer School \textit{Literary Studies \&
      Digital Humanities}, 31 May 2021, Leiden University.
  \item[2021] \tab{}``Automated Authorship Attribution''. Guest lecture at Leiden
    University, April 28 2021, Leiden University.     
  \item[2021] \tab{}``Cultural Evolution and the Humanities''. Guest lecture at Antwerp
    University, April 22, 2021. Antwerp University. 
  \item[2018] \tab{}``How to Read a Million Stories? Digital Text Analysis for the Study of
    Children's Literature''. Full-day tutorial at the Children’s Literature Summer School
    2018, July 1 -- 5, 2018, University of Antwerp.  
  \item[2017] \tab{}``Skills Training ’Introduction to Programming'', December 4 -- 8,
    2017, Doctoral Training Unit `Digital History \& Hermeneutics' (DHH), University of
    Luxembourg. 
  \item[2017] \tab{}``Python programming for the humanities and allied social sciences'',
    Mar 29, 2017 - Mar 31, 2017, Radboud Digital Humanities Spring School 2017
  \item[2017] \tab{}``Text Analysis with Python'', January 27, 2017, Digital Humanities
    Workshop Series, University of Manchester
  \item[2016] \tab{}``Scraping Twitter with Python''. Workshop on scraping Twitter data
    with Python. May 24, 2016. Digital Disruption in Asia Conference, Leiden. 
  \item[2016] \tab{}``Cultural Selection of Fairy Tales''. Guest lecture on cultural
    selection and evolution of fairy tales. April 28, 2016, Digital Humanities Lecture
    Series, Radboud University. 
  \item[2016] \tab{}``Digital Text Analysis'', Five-day (MA, PhD-level) course on
    computational text analysis, with a special focus on `authorship attribution', `text
    normalization', and `linguistic profiling', January 18 -- 22, 2016, LOT Winter School,
    Tilburg University.
  \item[2016] \tab{}``Python for the Humanities'', The objective of the full (MA,
    PhD-level) course was to familiarize students with the programming language Python for
    the computational processing of texts. December 2015 -- April 2016, Ghent University. 
  \item[2015] \tab{}``Python for the Arts and Humanities''. Four-day (MA, PhD-level)
    workshop on programming with Python for the Humanities. March 23 -- 26, 2015, Literary
    Lab, Ghent University.
  \item[2015] \tab{}``Python for the Arts and Humanities''. Three-day (MA, PhD-level)
    workshop on programming with Python in the Humanities and Social Sciences, February 16
    -- 18, 2015, Maynooth University
  \item[2014] \tab{}``Introduction and tutorial in Humanities Programming with Python''.
    Five-day (MA, PhD-level) workshop on programming with Python for the Humanities and
    Social Sciences. August 17 -- 30, 2014. University of Göttingen. 
  \item[2014] \tab{}``Advanced Programming with Python''. Five-day (MA, PhD-level)
    workshop on advanced topics in Humanities programming. European Summer School in
    Digital Humanities Culture \& Technology, July 22 -- August 1, 2014. 
  \item[2014] \tab{}``Introduction into Python for the Humanities and Social Sciences''.
    March 31 -- April 4 2014, University of Antwerp. 
  \item[2013] \tab{}``Introduction into Python Programming for the Humanities'', Three-day
    (MA-, PhD-level) workshop on programming with Python for the Humanities. With
    Maarten van Gompel. April 2013, Radboud University.
  \end{tablist}

\section*{Thesis Supervision}
\begin{tablist}
  \item[2022-] \tab{}Supervision PhD thesis by Arjan van Dalfsen, Utrecht University;
  \item[2023] \tab{}Supervision MA thesis by Jurrian Kooiman about detecting and
    classifying aspects of focalisation in literary texts, Utrecht University;
  \item[2021] \tab{}Supervision MA thesis by Arjan van Dalfsen on applying
    Unseen Species models to the Short Title Catalog, the Netherlands, Utrecht University;
  \item[2020] \tab{}Supervision Internship Liesje Linden about applying Neural Language
    Models to classify grammatical connectives in Dutch, Utrecht University;
  \item[2020] \tab{}Supervision MA Thesis Joris Veerbeek on a computational perspective to
    the evaluation of literary critics, Utrecht University;
  \item[2018] \tab{} Supervision MA thesis by Alie Lassche on Cultural Evolution
    in Dutch Early Modern Songs. Topical Fluctuations in the Dutch Song Database
    (1550-1750), Utrecht University;
\end{tablist}

\section*{Current Academic Service}
\begin{tablist}
\item[2021] \tab{}Guest editor of the \textit{Journal of Open Humanities Data}
\item[2021] \tab{}Programme Chair of the SciPy 2021 Symposium on Computational Social Science
  \& Digital Humanities
\item[2020 -- 2022] \tab{}Founder and Programme Chair of the \textit{Computational Humanities
    Research} conference
\item[2017 -- 2022] \tab{}Steering and Programme Committee member of the DH Benelux conference
\end{tablist}

\section*{Selected Media Coverage}
\begin{tablist}
\item[2022] \tab{}The big idea: could the greatest works of literature be undiscovered?
  \textit{Guardian}, Mey 30, 2022.
\item[2022] \tab{}How much medieval literature has been lost? \textit{Scientific
    American}, March 8, 2022.
\item[2022] \tab{}`Lost' medieval literature uncovered by techniques used to track
  wildlife, \textit{Science Magazine}, February 17, 2022.
\item[2022] \tab{}Seule une infime partie des récits médiévaux a survécu, \textit{Le
    Figaro}, March 11, 2022.
\item[2022] \tab{}Study finds 90 percent of medieval chivalric and heroic manuscripts have
  been lost, \textit{Ars Technica}, February 21, 2022.
\item[2022] \tab{}De helft van de Nederlandse ridderromans is waarschijnlijk voorgoed
  verdwenen, \textit{de Volkskrant}, February 18, 2022.
\item[2020] \tab{}De jacht op de verloren ridderromans. \textit{NRC}, August 24, 2020.
\item[2019] \tab{}Man (m/v) valt in put... Hebben alle succesvolle verhalen eigenlijk
  dezelfde onderliggende plot?, \textit{NRC}. July 13, 2019. 
\item[2018] \tab{}Taal is de sleutel tot echte artificiële intelligentie. \textit{De Tijd}.
  September 15, 2018.
\item[2018] \tab{}Raprobot krijgt op Lowlands een kans voor het grote publiek. \textit{Trouw}.
  August 17, 2018. 
\item[2018] \tab{}Rapper MC Turing heeft een indrukwekkende flow, voor een robot. \textit{de
    Volkskrant}, August 2018. 
\item[2017] \tab{}Guest on the television program \textit{Pauw} to talk about the artificial
  writer Asibot, October 31, 2017.
\item[2017] \tab{}Giphart schrijft samen met een robot. \textit{Trouw}, June 16, 2017. 
\item[2017] \tab{}Giphart krijgt hulp van deze literaire robot. \textit{NOS op 3}, June 16, 2017. 
\item[2017] \tab{}Robot schrijft verhaal, Giphart ondersteunt, \textit{EenVandaag}, July
  11, 2017.
\item[2016] \tab{}De grote boze wolf wordt steeds minder boos. \textit{de Volkskrant},
  July 4, 2016. 
\item[2016] \tab{}Pas op, er zit een moraal aan het verhaal. \textit{De Standaard}, June
  4, 2016. 
\item[2016] \tab{}New study could explain why we remake certain movies over and over
  again. Online article in Ars Technica, June 30, 2016.
\end{tablist}
  

\begin{center}
  \vfill
Updated \monthyeardate\today
\end{center}

\end{document}
